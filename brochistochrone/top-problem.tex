%선택지 문항옵션으로 블로그 `책읽기의 낙원'에서 얻은 방법
%선택지 문항 옵션
\usepackage{xparse}
\usepackage{kocircnum}
\usepackage{tabto}
\usepackage{environ}

\NewEnviron{myoptions}{\TreatOptions{\BODY}}

\newdimen\TabLen \TabLen=3cm
\TabPositions{0cm,\TabLen,2\TabLen}

\ExplSyntaxOn

\dim_new:N \l_check_op_len_dim
\dim_zero_new:N \g_op_len_dim
\int_new:N \op_choice_int

\cs_new:Npn \check_op_len:n #1
{
    \hbox_set:Nn \l_tmpa_box { #1 }
    \dim_set:Nn \l_check_op_len_dim { \box_wd:N \l_tmpa_box }
    \dim_compare:nT { \l_check_op_len_dim > \g_op_len_dim }
    {
        \dim_gset_eq:NN \g_op_len_dim \l_check_op_len_dim
    }
}

\cs_new:Npn \disp_op_choices:n #1
{
   \int_incr:N \op_choice_int
   \head_of_item
   {
       \hcrcircnum { \int_use:N \op_choice_int }
       \space\space
    }
    #1 \tab_here:
}

\NewDocumentCommand \TreatOptions { m }
{
    \clist_set:No \op_args_clist { #1 }
    \clist_map_function:NN \op_args_clist \check_op_len:n
    \dim_compare:nTF 
    { \g_op_len_dim > \dim_eval:n {\TabLen-10pt-4pt} }
    {
        \tl_set_eq:NN \head_of_item \hangfrom
        \cs_set:Nn \tab_here: {\par\noindent}
    }
    {
        \tl_clear:N \head_of_item
        \cs_set:Nn \tab_here: {\tab}
    }
    \par\smallskip\noindent
    \int_zero:N \op_choice_int
    \clist_map_function:NN \op_args_clist \disp_op_choices:n
}

\ExplSyntaxOff


%%연습문제환경을 만듦 

\usepackage{jiwonlipsum}
\usepackage{kswrapfig}

\newskip\mybestskip
\mybestskip=1\onelineskip plus \baselineskip minus .5\baselineskip %문제들 사이의 간견을 조금 좁게 만들고자, \mybestskip=4 에서 4를 1로 바꿈.

\newcounter{MCnum}
\newenvironment{myProb}%
{%
	\refstepcounter{MCnum}%
%	\noindent\llap{\mbox{{\Large\bfseries\arabic{section}}-{\sffamily\theMCnum}}\space}% 이게 원본 세팅이었으나 섹션넘버를 쓰지 않는 경우가 많을 것 같아서 주석처리하고 아래처럼 바꾸었다.
		\noindent\llap{\mbox{\Large{\theMCnum}.}\space}
	\ignorespaces
}%
{%
	\par
	\vskip\mybestskip
}
 %%%%%%%% 여기는 특별 문제의 번호 왼쪽 옆에 *를 붙이도록 하려는 시도이다.
\newenvironment{myProb*}%
{%
	\refstepcounter{MCnum}%
%	\noindent\llap{\mbox{{\Large\bfseries\arabic{section}}-{\sffamily\theMCnum}}\space}% 이게 원본 세팅이었으나 섹션넘버를 쓰지 않는 경우가 많을 것 같아서 주석처리하고 아래처럼 바꾸었다.
		\noindent\llap{\mbox{\Large{\(^{*}\)\theMCnum}.}\space}
	\ignorespaces
}%
{%
	\par
	\vskip\mybestskip
}




%%%%%%%%%여기까지가 새로운 시도였다.

%% to fix the bug(?) of counters in kswrapfig (or picins?)
\usepackage{etoolbox}
\pretocmd{\kswrapfig}{\addtocounter{MCnum}{-1}}{}{}