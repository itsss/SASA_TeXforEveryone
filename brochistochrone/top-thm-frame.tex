%%정리류 환경들로 KTUG의 progress님이 만든 내용들
%%% theorems

%%% written by progress
%%% 
%% 색깔 설정
% Pantone Fashion Color: Spring 2014
\definecolor{PlacidBlue}{cmyk}{.47,.17,.02,.0}
\definecolor{VioletTulip}{cmyk}{.44,.39,.0,.0}
\definecolor{Hemlock}{cmyk}{.39,.04,.35,.0}
\definecolor{Paloma}{cmyk}{.35,.24,.27,.0}
\definecolor{Sand}{cmyk}{.20,.27,.48,.0}
\definecolor{Freesia}{cmyk}{.0,.14,.100,.0}
\definecolor{Cayenne}{cmyk}{.06,.74,.56,.0}
\definecolor{CelosiaOrange}{cmyk}{.0,.63,.80,.0}
\definecolor{RadiantOrchid}{cmyk}{.32,.65,.0,.0}
\definecolor{DazzlingBlue}{cmyk}{.92,.57,.0,.0}
\definecolor{PurpleHaze}{cmyk}{.56,.51,.15,.0}
\definecolor{Comfrey}{cmyk}{.74,.28,.63,.10}
\definecolor{MagentaPurple}{cmyk}{.51,.94,.24,.24}

\colorlet{MainColorOne}{Comfrey}
\colorlet{MainColorTwo}{PurpleHaze}

%\DeclareRobustCommand{\myem}[2][\empty]{%
%\ifx#1\empty
%#2\index{#2}%
%\else
%#2\index{#2}(#1)\index{#1}%
%\fi}

\makeatletter
% \newtheorem{thm}{정리}[chapter]
% \newtheorem{cor}[thm]{보조정리}
% \renewcommand{\proofname}{증명}

\renewcommand\thesection{\arabic{section}}

\newcounter{thm} % 새 카운터를 만들고
\setcounter{thm}{0} % 0으로 초기화

\counterwithin{thm}{section}


\newenvironment{thm}[1][\empty]{% 옵션 변수 하나를 받을 예정
\refstepcounter{thm} % 이게 있어야 번호가 순차적으로 늘어나며 \label과 \ref 가능
\begin{framed}
\ifx#1\empty
\noindent{\bfseries\sffamily정리 \thethm}\quad
\else
\noindent{\bfseries\sffamily정리 \thethm\enskip ({\small#1})}\quad % 정리 [ ] 안에 있는 부가 설명을 괄호 ( ) 안에 표시
\fi
}{
\end{framed}
}

\colorlet{shadecolor}{MainColorOne!20}
\newenvironment{cor}[1][\empty]{%
   \refstepcounter{thm} % 앞에서 선언했으니깐 이것 필요 없어
    \begin{framed}
    \ifx\empty#1
    \noindent{\bfseries\sffamily 따름정리 \thethm}\quad % 정리의 카운터를 이어서 사용
    \else
    \noindent{\bfseries\sffamily 따름정리 \thethm\enskip ({\small#1})}\quad 
    \fi
}{%
    \end{framed}
}

\newenvironment{Lem}[1][\empty]{%
   \refstepcounter{thm} % 앞에서 선언했으니깐 이것 필요 없어
    \begin{framed}
    \ifx\empty#1
    \noindent{\bfseries\sffamily 보조정리 \thethm}\quad % 정리의 카운터를 이어서 사용
    \else
    \noindent{\bfseries\sffamily 보조정리 \thethm\enskip ({\small#1})}\quad 
    \fi
}{%
   \end{framed}
}


\renewcommand{\proofname}{\color{MainColorOne}{\bfseries\sffamily 증명}}
\renewcommand\qedsymbol{\color{MainColorOne!50}■} % QED 마크 변경

\renewenvironment{proof}[1][\empty]{%
\par%
  \pushQED{\qed} % \qedhere 쓰기 위한 장치 ①. from amsthm.sty
\begin{description}
\ifx#1\empty
\item[\bfseries\sffamily\color{MainColorOne}증명]
\else
\item[\bfseries\sffamily\color{MainColorOne}증명({\small#1})] \quad
\fi
}{\popQED  % \qedhere 쓰기 위한 장치 ②
\end{description}%
\@endpefalse%
\par%
}

%풀이 환경을 만드려는 시도
\renewcommand{\proofname}{\color{MainColorOne}{\bfseries\sffamily Solution}}
\newcommand{\qedwhite}{\hfill  \ensuremath{\Box}} % QED 마크 변경

\newenvironment{sol}[1][\empty]{%
\par%
  \pushQED{\qedwhite} % \qedhere 쓰기 위한 장치 ①. from amsthm.sty
\begin{description}
\ifx#1\empty
\item[\bfseries Solution]
\else
\item[\bfseries Solution({\small#1})] \quad
\fi
}{\popQED  % \qedhere 쓰기 위한 장치 ②
\end{description}%
%\@endpefalse%
\par%
}


%원래의 증명 환경을 재현하려는 시도
\renewcommand{\proofname}{\color{MainColorOne}{\bfseries\sffamily Proof}}
\newcommand{\qedblack}{\hfill ■} % QED 마크 변경

\newenvironment{Proof}[1][\empty]{%
\par%
  \pushQED{\qedblack} % \qedhere 쓰기 위한 장치 ①. from amsthm.sty
\begin{description}
\ifx#1\empty
\item[\bfseries Proof]
\else
\item[\bfseries Proof ({\small#1})] \quad
\fi
}{\popQED  % \qedhere 쓰기 위한 장치 ②
\end{description}%
%\@endpefalse%
\par%
}

% \textsc
%정의 환경을 만드려는 시도

\newenvironment{Def}[1][\empty]{%
    \begin{framed}
    \ifx\empty#1
    \noindent{\bfseries\sffamily 정의 }\quad % 정리의 카운터를 이어서 사용
    \else
    \noindent{\bfseries\sffamily 정의 \enskip ({\small#1})}\quad 
    \fi
}{%
    \end{framed}
}


%\colorlet{shadecolor}{MainColorOne!20}
%\newenvironment{Def}[1][\empty]{%
%    \begin{snugshade}
%    \ifx\empty#1
%    \noindent{\bfseries\sffamily 정의 }\quad % 정리의 카운터를 이어서 사용
%    \else
%    \noindent{\bfseries\sffamily 정의 \enskip ({\small#1})}\quad 
%    \fi
%}{%
%    \end{snugshade}
%}

\makeatother



%여기부턴 내가 그냥 추가한 환경들
%theorems ; theorem에 글자를 기울이지 않기 위해 definition환경에 Theorem 환경을 추가하였다. Lem도 마찬가지. %% cor는 progress의 환경과의 충돌을 피해 주석처리.

%%% ams 패키지에 포함된 환경류들을 적절히 편집함.
%%%환경류 : Theorem, Lem, Cor, prop, conjecture는 영어로 표시되는 환경으로 번호는 thm을 따름. (모두 글자를 바로 세우기 위해 definition스타일로 정의함.) 대소문자 구분에 유의!
%% defn, Prob는 번호를 붙이지 않은 영어로 표시되는 환경이고 물론 definition스타일.
%% excs는 연습문제로 definition스타일이지만 번호만 매겨진다.
%% rem, exmap, aside 는 remark 스타일이고, 이 중 rem과 aside는 번호가 붙지 않는다.

%\newtheorem{theorem}[thm]{Theorem}
%\newtheorem{lem}[theorem]{Lemma}
%\newtheorem{cor}[theorem]{Corollary} 
\theoremstyle{definition}
%\newtheorem{prop}[theorem]{Proposition}
%\newtheorem*{defn}{Definition}
%\newtheorem{Theorem}[thm]{Theorem}
%\newtheorem*{Prob}{Problem}
%\newtheorem{Lem}[thm]{Lemma}
%\newtheorem{Cor}[thm]{Corollary} 
%\newtheorem{prop}[thm]{Proposition}
%\newtheorem{conjecture}[thm]{Conjecture}
\newtheorem{excs}{}
\newtheorem{excs*}[excs]{}
%\newtheorem{example}{Example}



\theoremstyle{remark}
\newtheorem*{rem}{Remark}
%\newtheorem{examp}{Example}
%\newtheorem*{aside}{Aside}

\newcounter{examp} % 새 카운터를 만들고
\setcounter{examp}{0} % 0으로 초기화



\counterwithin{examp}{section}
\renewcommand{\theexamp}{\arabic{examp}}

\newenvironment{examp}[1][\empty]{% 옵션 변수 하나를 받을 예정
\refstepcounter{examp} % 이게 있어야 번호가 순차적으로 늘어나며 \label과 \ref 가능
\par%
\begin{description}
  \ifx#1\empty
  \item[\bfseries\sffamily 보기 \theexamp.] 
  \else
  \item[\bfseries\sffamily 보기 \theexamp.\enskip ({\small#1})] % 정리 [ ] 안에 있는 부가 설명을 괄호 ( ) 안에 표시
  \fi
}{
\end{description}
%\@endpefalse%
\par%
}




